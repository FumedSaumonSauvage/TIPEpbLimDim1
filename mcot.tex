%%%%%%%%%%%%%%%%%%%%%%%%%%%%%%%%%%%%%%%%%%%%%%%%%%%%%%%%%%%%%%%%%%%%%%%%%%%
%
% Template for a LaTex article in English.
%
%%%%%%%%%%%%%%%%%%%%%%%%%%%%%%%%%%%%%%%%%%%%%%%%%%%%%%%%%%%%%%%%%%%%%%%%%%%

\documentclass[10pt]{article}
\usepackage[utf8]{inputenc} 
%\usepackage[francais]{babel}

%Dimensions
\usepackage{geometry }
\geometry{left=2cm,right=2cm,top=3.3cm,bottom=3.3cm}

% AMS packages:
\usepackage{amsmath, amsthm, amsfonts}

\usepackage{hyperref}

% Theoremess
%-----------------------------------------------------------------


\newtheorem{theoreme}{Th{\'e}or{\`e}me}[section]
\newtheorem{theoremedefinition}[theoreme]{Th{\'e}or{\`e}me et d{\'e}finition}
\newtheorem{definitionproposition}[theoreme]{Définition et Proposition}
\newcommand{\theoremeautorefname}{Th{\'e}or{\`e}me}
\newtheorem{proposition}[theoreme]{Proposition}%[section]
\newcommand{\propositionautorefname}{Proposition}
\newtheorem{lemme}[theoreme]{Lemme}%[section]
\newcommand{\lemmeautorefname}{Lemme}
\newtheorem{corollaire}[theoreme]{Corollaire}%[section]
\newcommand{\corollaireautorefname}{Corollaire}
\newtheorem{definition}[theoreme]{D\'efinition} %[section]
\newcommand{\definitionautorefname}{D\'efinition}
\newtheorem{definitions}[theoreme]{D\'efinitions}%[section]
\newtheorem{exemple}[theoreme]{Exemple}%[section]
\newtheorem{exemples}[theoreme]{Exemples}%[section]
\newtheorem{remarque}[theoreme]{Remarque}%[section]
\newcommand{\remarqueautorefname}{Remarque}
\newtheorem{remarques}[theoreme]{Remarques}%[section]
\newtheorem{probleme}[theoreme]{Probl{\`e}me}%[section]
\newtheorem{exercice}[theoreme]{Exercice}%[section]


% Shortcuts.
% One can define new commands to shorten frequently used
% constructions. As an example, this defines the R and Z used
% for the real and integer numbers.
%-----------------------------------------------------------------
\newcommand{\R}{\mathbb R}
\newcommand{\C}{\mathbb C}
\newcommand{\N}{\mathbb N}
\newcommand{\Z}{\mathbb Z}
\newcommand{\espace}{\hspace{5mm}}

% Similarly, one can define commands that take arguments. In this
% example we define a command for the absolute value.
% -----------------------------------------------------------------
\newcommand{\abs}[1]{\left\vert#1\right\vert}

% Operators
% New operators must defined as such to have them typeset
% correctly. As an example we define the Jacobian:
% -----------------------------------------------------------------
\DeclareMathOperator{\Jac}{Jac}

%-----------------------------------------------------------------
\title{Problèmes aux limites en dimension 1} % votre titre
\author{Simon Hergott} % votre Prénom Nom


\begin{document}
	\maketitle
	
\section{Abstract}

In this documentary research article, we will study boundary values problems in one dimension, and eventually give some clues for the second dimension. Boundary value problems are defined by a differential equation and a set of boundaries the solution needs to meet.  Its resolution in one dimension can be done for functions of one or more variables (p-variables differential equation), and is a key element to solve many physics problems such as heat conduction in a metal rod.
To solve this problem, we will need to introduce some other concepts, such as the finite differences method to approximate the solutions of a differential equation, and Green's function to get solutions the less dependent possible to the problem.

\section{Mots-clés}

\begin{itemize}
	\item BVP (Boundary Value Problem :  \emph{Problème aux limites}
	\item Green's function : \emph{fonction de Green}
	\item Finite differences (method) : \emph{(méthode des) différences finies}
\end{itemize}
	
\section{Bibliographie commentée}

\quad Nous devrons premièrement définir le problème aux limites en dimension 1: ici, nous utiliserons la définition trouvée notamment dans \cite{1}:
\begin{equation}
-u''(x) = f(x) \forall x \in [0,1] \espace u(0) = u(1)  =0
\end{equation}
Plusieurs autres définitions sont possibles, comme notamment dans \cite{2} où on ajoute $k^2 u$ au membre de gauche de l'équation différentielle, ainsi que $x$ peut se trouver sur n'importe quel intervalle $[0,L]$ et n'est plus contraint à $[0,1]$.


\quad Dans la résolution du problème aux limites la méthode des différences finies est utilisée pour trouver numériquement une solution (ou plus précisément, l' \emph{approximer}). Nous passerons alors rapidement sur la méthode des différences finies, en définissant dans notre cas une \emph{grille} de points sur laquelle résoudre le problème.
Nous définirons cette grille suivant \cite{1} telle qu'un ensemble de points $(x_j)_{j = 0}^n$ répartis avec un \emph{pas} $h$ sur $[0,1]$. \\
Pour développer sur les différences finies, nous nous aiderons de \cite{3}, particulièrement compréhensible, ainsi que de \cite{5} éventuellement.

\quad La résolution du problème nous amènera à utiliser la fonction de Green : elle peut être définie comme 
\begin{equation}
G\relax(x, s) = \left\{
    \begin{array}{ll}
        s(1-x) \espace s\in [0,x] \\
        x(1-s) \espace s\in [x, 1]
    \end{array}
\right.
\end{equation}
avec $x,s \in [0,1]$. Nous utiliserons les définitions de \cite{1} et {2}, la seconde permettant de vérifier que $G$ est bien une solution du problème, et qu'elle possède toutes les propriétés nécessaires pour être cohérente dans ce contexte. Cette fonction est particulièreme,t intéressante, car en écrivant $u(x) = \int_0^1 G\relax(x, s) f\relax(s) ds$ nous pouvons rendre les solutions du problème dépendantes exclusivement de $G$ et $f$. Nous étudierons la fonction de Green sur son intervalle $[0,1]^2$ pour continuer la recherche de solutions: nous utiliserons alors \cite{6}.

\quad Éventuellement, nous passerons rapidement sur la résolution en dimension 2 du problème, présente dans \cite{1} et appliquée dans \cite{5}.

\quad Ce problème étant très utilisé en physique, nous passerons en revue quelques unes de ces applications à travers la conduction de la chaleur,  exemple très courant et présenté notamment dans \cite{5} et \cite{7}.
De plus, nous verrons aussi la déformation d'une corde élastique à travers les applications de l'analyse de Fourier de \cite{7}, ainsi que le problème elliptique et le problème hyperbolique, qui sont deux utilisation du problème aux limites avec des équations différentielles partielles \cite{6}.

\section{Répartition du travail}
\quad Comme vous l'aurez constaté en lisant l'auteur, je suis seul.\\
Suivre la progression du rapport : \href{https://github.com/FumedSaumonSauvage/TIPEpbLimDim1}{https://github.com/FumedSaumonSauvage/TIPEpbLimDim1}


	
\begin{thebibliography}{Bibliographie}
 
\bibitem{1} Alfio Quarteroni, Riccardo Sacco, Fausto Saleri.Méthodes Numériques, \emph{Springer}2007, p. 369--452.


\bibitem{2}G.F.Roach. Green's functions,\emph{Cambridge University Press}, 1982, p.1--8

\bibitem{3}Sebastien Tordeux, Victor Peron. Methode des différences finies,\emph{Université de Pau}, 2020, p.1--48

\bibitem{4}Dean G.Duffy. Studies in advanced mathematics - Green's functions with applications,\emph{Chapman \& Hall}, 2001, p.13--242

\bibitem{5}Dennis.G.Zill. Differential equations with Boundary-Value-Problems, \emph{Cengage}, 2013, p.471--484

\bibitem{6}Ivar Stakgold, Michael Holst.Green's Function and Boundary-value-Problems, \emph{Wiley}, 2011, p.471--484 %vérifier si on le garde celui ci

\bibitem{7}Murray R.Spiegel. Fourier Analysis with applications to boundary value problems,\emph{McGraw-Hill}, 1974, p.1--19 %Pas cité mais pas grave
  
 \end{thebibliography}
\end{document}